% Some commands used in this file
\newcommand{\package}{\emph}

\chapter{Introduction}

\section{Motivation}
To test an algorithm or approach such as semiclassical wavepackets \cite{B_bachelor_thesis} for solving the time dependent Schr\"odinger equation a python implementation \cite{waveblocksnd} is at most times sufficient. In case longer simulation times and/or solving higher dimensional problems is demanded the computation time becomes an important factor. As such there is a need for a time efficient implementation which is only doable in languages such as C or C++. A starting implementation in C++ for scalar wavepackets is already done \cite{libwaveblocks} but further improvement is desired. One aspect which has to be taken into account is the storage format of the data but also the speed of these io-operations. As storage format the well-know HDF(hierarchical data format) is most suitable for our case given that it is a binary file format with sufficient io speed. The current form to store data is dependent on a external source \cite{eigen3-hdf5}, which is sufficient for small projects where data is supervised by the user, which uses HDF5 format but doesn't fully use its capabilities. Furthermore the chosen hierarchy is also not compatible with the dual python implementation. In this project a fully compatible hierarchy is implemented in C++ without the usage of an external source. Also for testing simulations under both implementations a data test was implemented using the well-know Google testing framework.

\section{Background}

In quantum physics the most prominent problems are governed by the time-dependent
Schr\"odinger equation \ref{eq:basics_tdse_simple}

\begin{equation} \label{eq:basics_tdse_simple}
  i \hbar \frac{\partial}{\partial t} \Ket{\varphi} = H \Ket{\varphi}
\end{equation}

where $H$ is the Hamiltonian, $\varphi(x,t)$ represents the wave function dependent on position $x$ and
time $t$ and $\langle \varphi | \varphi  \rangle$ is the probability density of electrons. This equation can be reformulated in a semiclassical setting for nuclei as: \
\begin{equation}
\label{eq:tdse_sc_nuclei}
 i\hbar \partial_{t}\psi = \left( -\frac{\hbar^{2}}{2} \Delta_{x} + V(\vec{x}) \right) \psi\,.
\end{equation}
Nevertheless there are still much challenges involved to solve this equation \ref{eq:tdse_sc_nuclei}. One of this is the high dimensionality of this equation. A molecule with $N$ nuclei where each of them has three degrees of freedom results in $3N$ unknowns. For example the simple molecule $\mathrm{CO_{2}}$ has already $d=9$ degrees of freedom. Another challenge is the multiple scales governed by the small parameter $\hbar = \varepsilon^{2}$ in case of $\mathrm{CO_{2}}$ it results in $\hbar \approx 0.0058$. Also the actual solution has frequencies of order $1/\hbar$ which are hard to reproduce for small $\hbar$ on a finite uniform grid as required by a Fourier based approach. Further there is the problem of long time evolutions.\\

In this semiclassical setting \textit{Hagedorn} wavepackets with its operators as described in \cite{H_ladder_operators} is a viable tool to overcome these challenges. Not only will it be gridfree but also a spectral based method which results in fast approximations in space of localized wave-functions. Lastly it further allows highly oscillating functions.\\

%basic algorithmisch mit hagedorn 
%mit splitting methoden P Q p q klassische probagation 
%quantum corrections

%second generation algorithmus Grad + Hagedorn
%[62]
A basic algorithm of computing quantum mechanics with \textit{Hagedorn} is specified in \cite{FGL_semiclassical_dynamics}. In this algorithm splitting method is used to divide the problem into classical propagation of \{P,Q,p,q\} with quantum correction terms.
Further improvement was done in \cite{GH_convsemiclassical}.\\

%waveblocks python master arbeit R.Bour
%diversitaet
A diverse implementation was done in python \cite{B_master_thesis} where source code is available on github \cite{waveblocksnd}. This implementation further supports various propagators. Interesting applications of this code can be found in the project about tunneling dynamics \cite{GHJ_tunneling_spawning} and non-adiabatic transitions \cite{BGH_natac}.\\
%application of code
%tunneling dynamics Gradinaru
%non adiabatic 

%C++ beginning
As python is an interpreted language it is mostly not optimized for execution time. To circumvent this problem a C++ implementation is desired. Until now only the base functionality was transferred to C++. Hagedorn wavepackets was done in \cite{bt_michajab}, inner products in \cite{st_benedekv} and the potentials in \cite{bt_lionelm}. The current base implementation is available on github \cite{libwaveblocks}. In this project the goal was to enlarge the C++ code functionality with intelligent data serialization. This was done with the well-known HDF5 library to reach compatibility with the structure of the data of the two implementation. This further allowed to use the GoogleTest framework to write a test which enabled to check if two simulations with different implementations yield the same data.


%This is version \verb-v1.4- of the template.
%
%We assume that you found this template on our institute's website, so
%we do not repeat everything stated there.  Consult the website again
%for pointers to further reading about \LaTeX{}.  This chapter only
%gives a brief overview of the files you are looking at.
%
%\section{Features}
%\label{sec:features}
%
%The rest of this document shows off a few features of the template
%files.  Look at the source code to see which macros we used!
%
%The template is divided into \TeX{} files as follows:
%\begin{enumerate}
%\item \texttt{thesis.tex} is the main file.
%\item \texttt{extrapackages.tex} holds extra package includes.
%\item \texttt{layoutsetup.tex} defines the style used in this document.
%\item \texttt{theoremsetup.tex} declares the theorem-like environments.
%\item \texttt{macrosetup.tex} defines extra macros that you may find
%  useful.
%\item \texttt{introduction.tex} contains this text.
%\item \texttt{sections.tex} is a quick demo of each sectioning level
%  available.
%\item \texttt{refs.bib} is an example bibliography file.  You can use
%  Bib\TeX{} to quote references.  For example, read
%  \cite{bringhurst1996ets} if you can get a hold of it.
%\end{enumerate}
%
%
%\subsection{Extra package includes}
%
%The file \texttt{extrapackages.tex} lists some packages that usually
%come in handy.  Simply have a look at the source code.  We have
%added the following comments based on our experiences:
%\begin{description}
%\item[REC] This package is recommended.
%\item[OPT] This package is optional.  It usually solves a specific
%  problem in a clever way.
%\item[ADV] This package is for the advanced user, but solves a problem
%  frequent enough that we mention it. Consult the package's
%  documentation.
%\end{description}
%
%As a small example, here is a reference to the Section \emph{Features}
%typeset with the recommended \package{varioref} package:
%\begin{quote}
%  See Section~\vref{sec:features}.
%\end{quote}
%
%
%\subsection{Layout setup}
%
%This defines the overall look of the document -- for example, it
%changes the chapter and section heading appearance.  We consider this
%a `do not touch' area.  Take a look at the excellent \emph{Memoir}
%documentation before changing it.
%
%In fact, take a look at the excellent \emph{Memoir} documentation,
%full stop.
%
%
%\subsection{Theorem setup}
%
%This file defines a bunch of theorem-like environments.
%
%\begin{theorem}
%  An example theorem.
%\end{theorem}
%
%\begin{proof}
%  Proof text goes here.
%\end{proof}
%
%Note that the q.e.d.\ symbol moves to the correct place automatically
%if you end the proof with an \texttt{enumerate} or
%\texttt{displaymath}.  You do not need to use \verb-\qedhere- as with
%\package{amsthm}.
%
%\begin{theorem}[Some Famous Guy]
%  Another example theorem.
%\end{theorem}
%
%\begin{proof}
%  This proof
%  \begin{enumerate}
%  \item ends in an enumerate.
%  \end{enumerate}
%\end{proof}
%
%\begin{proposition}
%  Note that all theorem-like environments are by default numbered on
%  the same counter.
%\end{proposition}
%
%\begin{proof}
%  This proof ends in a display like so:
%  \begin{displaymath}
%    f(x) = x^2.
%  \end{displaymath}
%\end{proof}
%
%
%\subsection{Macro setup}
%
%For now the macro setup only shows how to define some basic macros,
%and how to use a neat feature of the \package{mathtools} package:
%\begin{displaymath}
%  \abs{a}, \quad \abs*{\frac{a}{b}}, \quad \abs[\big]{\frac{a}{b}}.
%\end{displaymath}
