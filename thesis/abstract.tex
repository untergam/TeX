\begin{abstract}
It is of general interest in physics/chemistry to find viable algorithms to solve the time dependent Schr\"odinger equation. A means to an end is the \textit{Hagendorn} wave packet for a semi-classical approach. With this as basis a python implementation was created. This implementation can be used to simulate different kind of starting conditions which can result in a lot of data. To efficiently store this data it is desirable to use binary formats, which can be easily compressed if needed. A well-known data binary format is HDF which stands for hierarchical data format. To further improve simulation speed the python implementation was used as a basis for a C++ implementation. This thesis explains the intrinsic functionality of using the HDF5 library interface in C++. This was done in such a way that the data from the C++ implementation has the same hierarchy as the python implementation. Also for further use a test was implemented with the well-known GoogleTest framework to easily compare possible data between these two implementations.
\end{abstract}
