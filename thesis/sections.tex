\chapter{HDF5 C++ Interface}
First of all HDF stands for hierarchical data format meaning this binary format is allowing to structure the internal objects after the users demand.
%\section{Example Section}
%\subsection{Example Subsection}
%\subsubsection{Example Subsubsection}
%\paragraph{Example Paragraph}
%\subparagraph{Example Subparagraph}
\section{Overview}
\section{Internal types}
\section{Internal states}
\section{H5File}
\section{Group}
\section{DataSet}
\section{DataType}
\section{DataSpace}
\section{PropList}
\subsection{DSetCreatePropList}
\section{Attribute}



\chapter{Writer Template}

\section{DataType Declaration}

\section{Setup write options}

\section{Setup write hierarchy}

\section{Prestructure}

\section{Selection}

\section{Transformation}

\section{Writing}

\section{Extension}

\section{Update}

\section{Poststructure}

\section{Inner workings in a picture}

\section{Usage in a simulation main file}


\chapter{Data Test}

\section{Introduction to GoogleTest}

\section{The Main C++ File}


\chapter{Conclusion}

