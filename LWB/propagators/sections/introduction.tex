\section*{Introduction}
%
With the growing understanding for quantum mechanics and dynamics, one can attempt to solve increasingly complex systems and computational problems.
Without doubt, time propagation is at the core of simulating quantum systems and getting a better understanding for them.
\par\medskip
%
The aim of this project was to give a summary of some state of the art quantum time propagators and provide an efficient C++ implementation for evolving Hagedorn wave packets in time.
The produced code is an addition to the C++ WaveBlocks framework whose functionality has already been extended in the context of preceding student projects.
The entire WaveBlocks project, which is partially based on an existing Python project with the same name, can be found at \cite{libwaveblocks}.
\par\medskip
%
Section \ref{sec:operatorsplitting} repeats some of the most important mathematical ideas underlying the time propagation of wave packets and introduces the notations that will be used throughout the rest of the document.
Based on these mathematical observations, section \ref{sec:buildingblocks} prepares a toolbox of essential building blocks that will help in the construction of propagators as well as for the final understanding of the source code.
In section \ref{sec:propagators}, a series of numerical propagators for quantum wave packets are presented and expressed in terms of the components provided earlier.
Some detailed insights into the most important technical ideas of the C++ implementation are given in section \ref{sec:implementation}.
Finally, section \ref{sec:results} presents the results of a series of numerical experiments to show the correctness of the implemented algorithms and analyze some performance aspects like the scaling with step size, dimensionality of the wave packet and order of the splitting coefficients.
