\chapter{Introduction}

\section{Motivation}
To test the reliability of a numerical method or an approximation, e.g. such as in \cite{B_bachelor_thesis}, an implementation in Python is sufficient. Python is easy to write and understand and the code can also be used as a base e.g. for a C++ version. A Python framework to solve the time-dependent Schr\"odinger equation with a semiclassical approach with \textit{Hagedorn} wavepackets is already available on github \cite{waveblocksnd}. This can be used for example to test different propagation methods, quadrature functions or integration schemes. In case longer simulations or solving higher dimensional problems is desired the execution time in Python gets unfeasible. Therefore it is necessary to port the Python implementation to C or C++ which are highly efficient. The core implementation is already available in C++ on github \cite{libwaveblocks} but further improvement is desired. This implementation currently uses an external project \cite{eigen3-hdf5} for writing data in \textit{HDF5} format. This project is sufficient if the user supervises the generated data. The goal of this project is to use the \textit{HDF5} interface \cite{hdf5cppdoc} directly to write data in a comparable way to Python. This further allows to implement a data test which can compare data files between Python and C++. This was done with the well-known \textit{GoogleTest} \cite{googletest} framework.

\section{Background}
\label{seq:background}

In quantum physics the most prominent problems are governed by the time-dependent
Schr\"odinger equation \ref{eq:basics_tdse_simple}

\begin{equation} \label{eq:basics_tdse_simple}
  i \hbar \frac{\partial}{\partial t} \Ket{\psi} = H \Ket{\psi}
\end{equation}

where $H$ is the Hamiltonian, $\psi(\vec{x},t)$ represents the complex valued wave function dependent on position $\vec{x} \in \mathbb{R}^{d}$ and
time $t$ and $\langle \psi | \psi  \rangle$ is the probability density of electrons. This can be found in the most basic courses and books about quantum physics. This equation can be rescaled in a semiclassical setting to obtain the nuclear Schr\"odinger equation: \
\begin{equation}
\label{eq:tdse_sc_nuclei}
 i\hbar \partial_{t}\psi = \left( -\frac{\hbar^{2}}{2} \Delta_{x} + V(\vec{x}) \right) \psi\,.
\end{equation}

The goal is to find for $\psi = \psi(\vec{x},t)$ depending on the spatial variables $\vec{x} = (x_{1},\ldots,x_{d}) \in \mathbb{R}^{d}$ and the time $t\in \mathbb{R}$. In this context $\Delta_{x}$ is the Laplace operator and $V$ is a smooth real-valued potential, e.g. an electronic energy surface in the time-dependent Born-Oppenheimer approximation. Nevertheless there are still many challenges involved in solving this equation \ref{eq:tdse_sc_nuclei}. One of these is the high dimensionality of this equation. A molecule with $N$ nuclei where each of them has three degrees of freedom results in $3N$ coordinates. For example the simple molecule $\mathrm{CO_{2}}$ has already $d=9$ degrees of freedom. Another challenge is the multiple scales governed by the small parameter $\hbar = \varepsilon^{2}$ in case of $\mathrm{CO_{2}}$ it results in $\hbar \approx 0.0058$. Also the actual solution has frequencies of order $1/\hbar$ which are hard to reproduce for small $\hbar$ on a finite uniform grid as required the widely used Fourier based approach. Further there is the problem of long time evolution.\\

The semiclassical method makes use of the raising and lowering operators for semiclassical wavepackets \cite{H_ladder_operators}. The raising and lowering operators $\mathcal{R}$ and $\mathcal{L}$ produce a complete $L^2$-orthonormal set of basis functions $\phi_{\vec{k}}(\vec{x}) = \phi_{\vec{k}}^\hbar[\vec{q},\vec{p},\mat{Q},\mat{P}](\vec{x})$ (for multi-indices $\vec{k} = (k_1,\dots,k_d)$ with non-negative integers $k_j$) obeying the three-term recurrence:
\begin{equation*}
          \left(\sqrt{k_{j}+1}\phi_{\vec{k}+\vec{e^j}}\right)_{j=1}^{d}
          = \sqrt{\frac{2}{\hbar}} \mat{Q}\inv \left(\vec{x}-\vec{q}\right) \phi_{\vec{k}} -
          \mat{Q}\inv \conj{\mat{Q}} \left(\sqrt{k_{j}} \phi_{\vec{k}-\vec{e^j}}\right)_{j=1}^{d} \, .
\end{equation*}

Therefore we have Hagedorn wavepackets of the following form:
        \begin{equation*}
          \begin{split}
            \phi_{\vec{0}}^\hbar & [\vec{q},\vec{p},\mat{Q},\mat{P}](\vec{x})\ =
            (\pi\hbar)^{-\frac{d}{4}} (\det \mat{Q})^{-\frac{1}{2}} \,\times \\
            & \exp\left(\frac{i}{2\hbar} \left(\vec{x}-\vec{q}\right)\T \mat{P}\mat{Q}\inv \left(\vec{x}-\vec{q}\right) +
              \frac{i}{\hbar} \vec{p}\T \left(\vec{x}-\vec{q}\right) \right)
          \end{split}
        \end{equation*}
where $\vec{q}\in \mathbb{R}^d$ and $\vec{p}\in \mathbb{R}^d$ represent the position and momentum, respectively, and $\mat{Q}$ and $\mat{P}$ are complex $d\times d$ matrices satisfying some compatibility relations. For a detailed explanation see \cite{FGL_semiclassical_dynamics}. In \cite{FGL_semiclassical_dynamics} is also a basic time propagation algorithm described. Further improvement and a proof about convergence of semiclassical wavepackets based time-splitting was presented in \cite{GH_convsemiclassical}.\\

This method was implemented by R. Bourquin in his master thesis \cite{B_master_thesis} in Python and is available on github \cite{waveblocksnd}. This code can be used in applications for example tunneling dynamics and spawning \cite{GHJ_tunneling_spawning} or non-adiabatic transition near avoided crossings \cite{BGH_natac}. Until now only the core of this implementation was already ported to C++. Hagedorn wavepackets was done by M. B\"osch \cite{bt_michajab}, inner products and quadrature by B. Vartok \cite{st_benedekv} and matrix potentials by L. Miserez \cite{bt_lionelm}. This core implementation is available on github \cite{libwaveblocks}. In this project the goal was to enlarge the C++ code functionality with a more useful data serialization. The goal included compatibility between the Python and C++ implementation on how data is stored using the \textit{HDF5} format. This further allowed to use the \textit{GoogleTest} framework to write a data test for the two \textit{HDF5} files. This can be convenient because now we can test if both implementation generate the same data or if there could be a bug in the code.