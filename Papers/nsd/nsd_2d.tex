\documentclass[a4paper,10pt]{article}
\usepackage[utf8]{inputenc}
\usepackage{amsmath}
\usepackage{amssymb}
\usepackage{amsfonts}

\usepackage{color}
\definecolor{linkcol}{rgb}{0,0,0.4}
\definecolor{citecol}{rgb}{0.5,0,0}

\usepackage[pagebackref,hyperindex=true]{hyperref}
\hypersetup{colorlinks=true,linkcolor=linkcol,citecolor=citecol,urlcolor=linkcol}

\usepackage{amsmath}
\usepackage{amssymb}
\usepackage{amsfonts}
\usepackage{amsopn}
\usepackage{braket}
\usepackage{bbm}
\usepackage{dsfont}
% \usepackage{mathabx}


% Various new commands that ease typesetting math even further
% \newcommand{\assign}{\ensuremath{\coloneq}}
% \newcommand{\rassign}{\ensuremath{\eqcolon}}
\newcommand{\assign}{\ensuremath{:=}}
\newcommand{\rassign}{\ensuremath{=:}}
\newcommand{\seteq}{\ensuremath{\overset{!}{=}}}

\newcommand{\of}[1]{\ensuremath{\left( #1 \right)}}
\newcommand{\ofs}[1]{\ensuremath{\left( #1 \right)}}

%\newcommand{\norm}[1]{\ensuremath{\| #1 \|}}

\newcommand{\tmop}[1]{\ensuremath{\operatorname{#1}}}

\newcommand{\id}{\ensuremath{\mathds{1}}}
% \newcommand{\id}{\ensuremath{I}}

\newcommand{\kron}[1]{\ensuremath{\delta_{#1}}}

\newcommand{\conj}[1]{\ensuremath{\overline{#1}}}

\renewcommand{\vec}[1]{\ensuremath{\underline{#1}}}
\newcommand{\mat}[1]{\ensuremath{\mathbf{#1}}}

\newcommand{\inv}{\ensuremath{{}^{-1}}}
\newcommand{\T}{\ensuremath{{}^{\textnormal{T}}}}
\renewcommand{\H}{\ensuremath{{}^{\textnormal{H}}}}
\newcommand{\Tinv}{\ensuremath{{}^{\textnormal{-T}}}}
\newcommand{\Hinv}{\ensuremath{{}^{\textnormal{-H}}}}

\newcommand{\tr}{\ensuremath{\textnormal{Tr}}}

\newcommand{\ft}[1]{\ensuremath{\mathcal{F}\left(#1\right)}}
\newcommand{\ift}[1]{\ensuremath{\mathcal{F}^{-1}\left(#1\right)}}

\newcommand{\fft}[1]{\ensuremath{\mathtt{FFT}\left(#1\right)}}
\newcommand{\ifft}[1]{\ensuremath{\mathtt{IFFT}\left(#1\right)}}

\newcommand{\dotp}[2]{\ensuremath{\left\langle #1 , #2 \right\rangle}}

\newcommand{\bigO}[1]{\ensuremath{\mathcal{O}\left( #1 \right)}}

\newcommand{\laplace}{\ensuremath{\operatorname{\Delta}}}

\newcommand{\di}[1]{\ensuremath{\mathrm{d}#1}}
\newcommand{\diff}[3][]{\frac{\mathrm{d}^{#1}#2}{\mathrm{d}#3^{#1}}}
\newcommand{\pdiff}[2]{\frac{\partial #1}{\partial #2}}
\newcommand{\pdiffn}[3]{\frac{\partial^{#1}#2}{\partial #3^{#1}}}

% EOF


%opening
\title{Numerical Steepest Descent for Wavepackets}
\author{R. Bourquin}

\parindent 0pt

\begin{document}

\maketitle

\section{Primary task}

Given functions $f(x_1, \ldots, x_n):\mathbb{R^N}\rightarrow\mathbb{R}$ and
$g(x_1, \ldots, x_n):\mathbb{R^N}\rightarrow\mathbb{C}$, compute the integral:

\begin{equation} \label{eq:hoi}
 I := \int_{-\infty}^{\infty} \cdots \int_{-\infty}^{\infty} f \exp(i \omega g) \, \mathrm{d}x_1 \cdots \mathrm{d}x_n
\end{equation}

This is a highly oscillatory integral. Usually we call $g$ the \emph{oscillator}
and $f$ the \emph{amplitude} or \emph{enveloppe}. For more details see the work
by Huybrechs and some others. The main technique is described in \cite{HV_hoq} for
the one-dimensional case and in \cite{HV_cub} for multivariate integrands.
Semi-infinite intervalls are treated in \cite{H_nsd_sii}.


\section{Numerical steepest descent}

The goal is to find a transformation $u_i(\cdot)$ of variables $x_i$ into new variables $q_i$
and rewrite the above integral \eqref{eq:hoi} in the form:

\begin{equation} \label{eq:noi}
 \begin{split}
 I^\prime = \exp(i \omega g(\vec{x}^{*}))
     \int_{0}^{\infty} \cdots \int_{0}^{\infty}
     f\left(u_1\left(q_1\right), u_2\left(u_1\left(q_1\right), q_2\right), \ldots, u_N\left(\cdot\right)\right) \\
     \pdiff{u_1\left(q_1\right)}{q1}
     \pdiff{u_2\left(u_1\left(q_1\right), q_2\right)}{q_2}
     \cdots
     \pdiff{u_N\left(\cdot\right)}{q_N}
     \mathrm{d}q_1 \cdots \mathrm{d}q_N
 \end{split} \,.
\end{equation}

The transformed integrand is no longer oscillatory. The integral $I^\prime$
can then be computed easily by (generalized) Gauss-Laguerre quadrature rules.

Since the nested structure of the functions $u_i$ is rather confusing,
we would like to find a simpler denested representation:

\begin{equation} \label{eq:noi2}
 \begin{split}
 I^{\prime\prime} = \exp(i \omega g(\vec{x}^{*}))
     \int_{0}^{\infty} \cdots \int_{0}^{\infty}
     f\left(\tilde{u}_1\left(q_1\right), \tilde{u}_2\left(q_1, q_2\right), \ldots, \tilde{u}_N\left(q_1, \ldots, q_N\right)\right) \\
     \pdiff{\tilde{u}_1\left(q_1\right)}{q1}
     \pdiff{\tilde{u}_2\left(q_1, q_2\right)}{q_2}
     \cdots
     \pdiff{\tilde{u}_N\left(q_1, \ldots, q_N\right)}{q_N}
     \mathrm{d}q_1 \cdots \mathrm{d}q_N
 \end{split} \,.
\end{equation}


\section{Generalized Gauss-Laguerre quadrature rules}

In one dimension the nodes $x_i$ of the (generalized) Gauss-Laguerre quadrature
rule of order $n$ are given by the roots of the Laguerre polynomial $L^{\alpha}_n(x)$.
(Non-generalized case is obtained by setting $\alpha = 0$.) The weights are given by:

\begin{equation}
 w_i := \frac{-\Gamma(\alpha+n)}{n \Gamma(n) L^{\alpha}_{n-1}(x_i) L^{\alpha+1}_{n-1}(x_i)}
\end{equation}

For multi-dimensional quadrature we can use a tensor product ansatz:

\begin{equation}
 X_i = \bigotimes_{k=1}^N x_i \quad\quad
 W_i = \prod_{k=1}^N w_i
\end{equation}

Given an integral of the form:

\begin{equation}
 J := \int_{0}^{\infty} \cdots \int_{0}^{\infty} h(t_1, \ldots, t_N) \exp(-\omega t_1) \cdots \exp(-\omega t_N)
 \, \mathrm{d}t_1 \cdots \mathrm{d}t_N
\end{equation}

we first transform it to:

\begin{equation}
 J = \frac{1}{\omega^N}
 \int_{0}^{\infty} \cdots \int_{0}^{\infty} h\left(\frac{t_1}{\omega}, \ldots, \frac{t_N}{\omega}\right)
 \exp(-t_1) \cdots \exp(-t_N) \, \mathrm{d}t_1 \cdots \mathrm{d}t_N
\end{equation}

and then apply the quadrature rule:

\begin{equation}
 J \approx Q := \frac{1}{\omega^N} \sum_{k_1}^n \cdots \sum_{k_N}^n
                h\left(\frac{x_{k_1}}{\omega}, \ldots, \frac{x_{k_N}}{\omega}\right)
                w_{k_1} \cdots w_{k_N}
\end{equation}

In case $h$ is (weakly) singular for all $t_i \rightarrow 0$ we have to regularize
the integrand by using \emph{generalized} Gauss-Laguerre quadrature. In our case
we have:

\begin{equation}
 h = \frac{\tilde{h}(t_1,\ldots,t_N)}{\sqrt{t_1}\cdots\sqrt{t_N}}
\end{equation}

and therefore choose $\alpha = -\frac{1}{2}$. The quadrature then looks like:

\begin{equation}
 Q := \frac{1}{\omega^N} \sum_{k_1}^n \cdots \sum_{k_N}^n
      h\left(\frac{x_{k_1}}{\omega}, \ldots, \frac{x_{k_N}}{\omega}\right)
      x_{k_1}^{-\alpha} \cdots x_{k_N}^{-\alpha}
      w_{k_1} \cdots w_{k_N}
\end{equation}

In our application $h$ will be the integrand of \eqref{eq:noi2} and because
of the derivatives therein the singularity appears.


\section{The case of a quadratic oscillator}

In our case the oscillator function $g$ is \emph{always} quadratic:

\begin{equation}
 g(\vec{x}) := \vec{x}\T \mat{M} \vec{x} + \vec{b}\T \vec{x} +c \,.
\end{equation}

In case $N=2$ we then have:

\begin{equation}
  \mat{M} :=
 \begin{pmatrix}
  a_{1,1} & a_{1,2} \\
  a_{2,1} & a_{2,2}
 \end{pmatrix}
 \quad
 \vec{b} :=
 \begin{pmatrix}
  b_1 \\ b_2
 \end{pmatrix}
 \quad \mathrm{and} \quad
 \vec{x} =
 \begin{pmatrix}
  x_1 \\ x_2
 \end{pmatrix}
\end{equation}

where in three dimensions:

\begin{equation}
  \mat{M} :=
 \begin{pmatrix}
  a_{1,1} & a_{1,2} & a_{1,3} \\
  a_{2,1} & a_{2,2} & a_{2,3} \\
  a_{3,1} & a_{3,2} & a_{3,3} \\
 \end{pmatrix}
 \quad
 \vec{b} :=
 \begin{pmatrix}
  b_1 \\ b_2 \\ b_3
 \end{pmatrix}
 \quad \mathrm{and} \quad
 \vec{x} =
 \begin{pmatrix}
  x_1 \\ x_2 \\ x_3
 \end{pmatrix} \,.
\end{equation}

This special form of $g$ makes the whole process of numerical steepest
descent much simpler although it still remains rather involved.


\section{Stationary points}

The gradient of $g$ is defined as usual:

\begin{equation}
 \vec{g} := \nabla g =
 \begin{pmatrix}
  \pdiff{g}{x_1} \\
  \vdots \\
  \pdiff{g}{x_N}
 \end{pmatrix}
 =
  \begin{pmatrix}
  g_1(x_1, \hdots, x_N) \\
  \vdots \\
  g_N(x_1, \hdots, x_N)
 \end{pmatrix}
\end{equation}

We can compute every component as:

\begin{equation}
 \begin{split}
  \pdiff{g}{x_i}  = \pdiff{\vec{x}\T \mat{M} \vec{x} + \vec{b}\T \vec{x} +c}{x_i}
                  = \pdiff{\vec{x}\T \mat{M} \vec{x}}{x_i} + \pdiff{\vec{b}\T \vec{x}}{x_i}
                  = \left[(\mat{M}+\mat{M}\T) \vec{x}\right]_{i} + \vec{b}_i
 \end{split}
\end{equation}

which is equivalent to:

\begin{equation}
 \pdiff{g}{x_i} = \sum_{k=1}^N \left(\mat{M}_{k,i}+\mat{M}_{i,k}\right)\vec{x}_k + \vec{b}_i
\end{equation}

Solving this for $x_i$ gives us a \emph{function} $x_i^{**}(x_1, \ldots, x_{i-1}, x_{i+1}, \ldots, x_N)$ like:

\begin{equation}
 x_i^{**} = \frac{\sum_{k=1}^{i-1} \left(\mat{M}_{k,i}+\mat{M}_{i,k}\right)\vec{x}_k
                  + \sum_{k=i+1}^{N} \left(\mat{M}_{k,i}+\mat{M}_{i,k}\right)\vec{x}_k
                  + \vec{b}_i  }
                 {-2 \mat{M}_{i,i}}
\end{equation}

The whole gradient can be computed at once by:

\begin{equation}
 \vec{g} = \nabla g = \pdiff{g(\vec{x})}{\vec{x}}
         = \pdiff{\vec{x}\T \mat{M} \vec{x}}{x} + \pdiff{\vec{b}\T \vec{x}}{x}
         = \mat{M} \vec{x} + \mat{M}\T \vec{x} + \vec{b}
\end{equation}

Next, we solve the following system:

\begin{equation}
 \vec{g} =
 \begin{pmatrix}
  g_0 \\ \vdots \\ g_N
 \end{pmatrix}
 = \left(\mat{M} + \mat{M}\T\right) \vec{x} + \vec{b}
 = \vec{0}
\end{equation}

for $x_i$ and call the solutions $x_i^{*}$ and gather them in a vector $\vec{x}^{*}$.
Formally:

\begin{equation}
 \vec{x}^{*} := -\left(\mat{M} + \mat{M}\T\right)\inv \vec{b}
\end{equation}

The solution is unique and well defined. Especially the $x_i^{*}$ are numbers
and do not depend on any parameters, in contrast to the $x_i^{**}$.
We know that the identity:

\begin{equation} \label{eq:denest_sp}
 x_i^{**}(x_1^{*}, \ldots, x_{i-1}^{*}, x_{i+1}^{*}, \ldots, x_N^{*}) = x_i^{*}
\end{equation}

holds.


\section{Finding the integration paths}

For each \emph{transformation} or \emph{integration path} we have to solve
a system of equations (called the \emph{path equation}). If we think of the
original integral \eqref{eq:hoi} as a nested onion like structure it seems
natural to begin with the last variable $q_N$ and searching for a path $u_N(\cdot)$
first.


\subsection{Two-dimensional case}

In the two-dimensional case the path equation reads:

\begin{equation} \label{eq:pequ2}
 g\left(
 \begin{bmatrix}
  x_1 \\ u_2(x_1, q_2)
 \end{bmatrix}
 \right)
 =
 g\left(
 \begin{bmatrix}
  x_1 \\ x_2^{**}(x_1)
 \end{bmatrix}
 \right)
 + i q_2
\end{equation}

and we try to solve this for $u_2(x_1, q_2)$. This equation has a \emph{free variable} $x_1$
and at this point the solution $u_2$ still depends on that free variable\footnote{This is
one of the main obstacles for an algorithmic and non-symbolic implementation.}.
Once we solved this system we can proceed with the transformation for $x_1$ and
write the other path equation as:

\begin{equation}
 g\left(
 \begin{bmatrix}
  u_1(q_1) \\
  x_2^{**}\left( u_1(q_1) \right)
 \end{bmatrix}
 \right)
 =
 g\left(
 \begin{bmatrix}
  x_1^{*} \\ x_2^{**}(x_1^{*})
 \end{bmatrix}
 \right)
 + i q_1
\end{equation}

We can simplify this further by applying the identity \eqref{eq:denest_sp} and get:

\begin{equation} \label{eq:pequ1}
 g\left(
 \begin{bmatrix}
  u_1(q_1) \\
  x_2^{**}\left( u_1(q_1) \right)
 \end{bmatrix}
 \right)
 =
 g\left(
 \begin{bmatrix}
  x_1^{*} \\ x_2^{*}
 \end{bmatrix}
 \right)
 + i q_1
\end{equation}

Solving this we obtain a transformation $u_1(q_1)$, This function
does not depend on any parameters beside $q_1$.

Now we can use the solution $u_1(q_1)$ of this system for replacing
$x_1$ in the solution $u_2$ above:

\begin{equation}
 \tilde{u}_2\left(q_1, q_2\right) := u_2\left( u_1\left(q_1\right), q_2 \right)
\end{equation}

Trivially we have $\tilde{u}_1 \equiv u_1$. In the end this gives us the
\emph{steepest descent manifold} $\mathcal{M}$ as:

\begin{equation}
\mathcal{M}(q_1, q_2) :=
 \begin{cases}
  \tilde{u}_1(q_1) \\
  \tilde{u}_2(q_1, q_2)
 \end{cases}
\end{equation}

from the defining equation \eqref{eq:pequ1} and \eqref{eq:pequ2}.

Note that the solution of these equations is usually not unique. In the case of $g$ beeing
quadratic we obtain two solutions $u_1^a$ and $u_1^b$ and two solutions $u_2^a$ and $u_2^b$.
We can use both, $u_1^a$ and $u_1^b$, for constructing $\tilde{u}_2$ and get four paths
$\tilde{u}_2^{a,a}$, $\tilde{u}_2^{a,b}$, $\tilde{u}_2^{b,a}$ and $\tilde{u}_2^{b,b}$.
In general we get $2^N$ steepest descent manifolds. This may make the method unsuitable
for larger $N$. (We drop the tilde from now on again.)


\subsection{Three-dimensional case}

When working in 3 space dimensions we have to solve three path equations.
The first one reads:

\begin{equation} \label{eq:pequ2}
 g\left(
 \begin{bmatrix}
  x_1 \\ u_2(x_1, q_2)
 \end{bmatrix}
 \right)
 =
 g\left(
 \begin{bmatrix}
  x_1 \\ x_2^{**}(x_1)
 \end{bmatrix}
 \right)
 + i q_2
\end{equation}


\subsection{Arbitrary dimensional case}

\marginpar{I'm really NOT sure if this section is at all correct!!}

In the general case of $N>2$ we solve iteratively $N$ different path equations
to get in turn $u_N$, $u_{N-1}$ up to $u_1$. First we start with:

\begin{equation} \nonumber
 g\left(
 \begin{bmatrix}
  x_1 \\ \vdots \\ x_{N-1} \\ u_N(x_1, \ldots, x_{N-1}, q_N)
 \end{bmatrix}
 \right)
 = g\left(
 \begin{bmatrix}
  x_1 \\ \vdots \\ x_{N-1} \\ x_N^{**}(x_1, \ldots, x_{N-1})
 \end{bmatrix}
 \right)
 +i q_N
\end{equation}

this gives us $u_N:\mathbb{R}^{N} \rightarrow \mathbb{C}$ as:

\begin{equation} \nonumber
 u_N\left(
 \begin{bmatrix}
  x_1 \\ \vdots \\ x_{N-1} \\ q_N
 \end{bmatrix}
 \right) = \cdots \,.
\end{equation}

Second we solve:

\begin{equation} \nonumber
 g\left(
 \begin{bmatrix}
  x_1 \\ \vdots \\ x_{N-2} \\ u_{N-1}(x_1, \ldots, x_{N-2}, q_{N-1}) \\ x_N^{**}(x_1, \ldots, x_{N-2}, u_{N-1}(\cdot))
 \end{bmatrix}
 \right)
 = g\left(
 \begin{bmatrix}
  x_1 \\ \vdots \\ x_{N-2} \\ x_{N-1}^{**}(x_1, \ldots, x_{N-2}, x_N^{*}) \\ x_{N}^{**}(x_1, \ldots, x_{N-2}, x_{N-1}^{*})
 \end{bmatrix}
 \right)
 +i q_{N-1}
\end{equation}

which gives us $u_{N-1}:\mathbb{R}^{N-1} \rightarrow \mathbb{C}$ as

\begin{equation} \nonumber
 u_{N-1}\left(
 \begin{bmatrix}
  x_1 \\ \vdots \\ x_{N-2} \\ q_{N-1}
 \end{bmatrix}
 \right) = \cdots \,.
\end{equation}

Third we solve:

\begin{equation} \nonumber
 g\left(
 \begin{bmatrix}
  x_1 \\ \vdots \\ x_{N-3} \\
  u_{N-2}(x_1, \ldots, x_{N-3}, q_{N-2}) \\
  x_{N-1}^{**}(x_1, \ldots, x_{N-3}, u_{N-2}(\cdot), u_{N}(\cdot)) \\
  x_N^{**}(x_1, \ldots, u_{N-2}(\cdot), u_{N-1}(\cdot))
 \end{bmatrix}
 \right)
 = g\left(
 \begin{bmatrix}
  x_1 \\ \vdots \\ x_{N-3} \\
  x_{N-2}^{**}(x_1, \ldots, x_{N-3}, x_{N-1}^{*}, x_N^{*}) \\
  x_{N-1}^{**}(x_1, \ldots, x_{N-3}, x_{N-2}^{*}, x_N^{*}) \\
  x_{N}^{**}  (x_1, \ldots, x_{N-3}, x_{N-2}^{*}, x_{N-1}^{*})
 \end{bmatrix}
 \right)
 +i q_{N-2}
\end{equation}

which gives us $u_{N-2}:\mathbb{R}^{N-2} \rightarrow \mathbb{C}$ as:

\begin{equation} \nonumber
 u_{N-2}\left(
 \begin{bmatrix}
  x_1 \\ \vdots \\ x_{N-3} \\ q_{N-2}
 \end{bmatrix}
 \right) = \cdots \,.
\end{equation}

This continues the same way until we arrive at the last equation:

\begin{equation} \nonumber
 g\left(
 \begin{bmatrix}
  u_1(q_1) \\
  x_2^{**}(u_1(\cdot), u_3(\cdot), \ldots, u_{N}(\cdot)) \\
  \vdots\\
  x_N^{**}(u_1(\cdot), \ldots, u_{N-1}(\cdot))
 \end{bmatrix}
 \right)
 = g\left(
 \begin{bmatrix}
  x_1^{**}(x_2^{*}, \ldots, x_{N}^{*}) \\
  x_2^{**}(x_1^{*}, x_3^{*}, \ldots, x_{N}^{*}) \\
  \vdots
  \\ x_{N}^{**}(x_1^{*}, \ldots, x_{N-1}^{*})
 \end{bmatrix}
 \right)
 +i q_1
\end{equation}

from where finally we obtain $u_1:\mathbb{R} \rightarrow \mathbb{C}$:

\begin{equation} \nonumber
 u_1\left(
 \begin{bmatrix}
  q_1
 \end{bmatrix}
 \right) = \cdots \,.
\end{equation}

\marginpar{Work out a 3D example to see if it fits the scheme}

Next we can combine the $u_i$ to obtain one after the other all denested $\tilde{u}_i$:

\begin{equation}
\begin{split}
 \tilde{u}_1(q_1)              & := u_1(q_1) \\
 \tilde{u}_2(q_1, q_2)         & := u_2(u_1(q_1), q_2) \\
 \tilde{u}_3(q_1, q_2, q_3)    & := u_3(u_1(q_1), u_2(u_1(q_1), q_2), q_3) \\
                               & \vdots \\
 \tilde{u}_N(q_1, \ldots, q_N) & := u_N(\cdots)
\end{split}
\end{equation}

At the end of the day we write our \emph{steepest descent manifold} $\mathcal{M}$ again as:

\begin{equation}
\mathcal{M}(q_1, \ldots, q_N) :=
 \begin{cases}
  \tilde{u}_1(q_1) \\
  \tilde{u}_2(q_1, q_2) \\
  \cdots \\
  \tilde{u}_N(q_1, q_2, \ldots, q_N) \\
 \end{cases}
\end{equation}

Even in case this is correct, the whole procedure is overly complicated and
it's questionable if it indeed can be applied (efficiently) to our integration problem.


\section{Sign issues}

From each tuple $\left(u_1^{j}, u_2^{l}\right)$ we get a contribution $Q_{j,l}$ according
to \eqref{eq:noi2} to the final value $Q \approx I$. But there is one small problem. We can
not simply add up all individual contributions $Q_{j,l}$. This would lead to wrong results
because some paths are incoming and some are outgoing and we must respect this orientation.

As a possible solution(?) we can compute a sign $\sigma_{k}$ for each path $u_1^k$:

\begin{equation}
 \sigma_1^k := sign( \Re(u_1^k(1) - x_1^{*}))
\end{equation}

and similarily for $u_2^{k,l}$:

\begin{equation}
 \sigma_2^{k,l} := sign( \Re(u_2^{k,l}(1,1) - x_2^{*})) \,.
\end{equation}

In the end we find (for a two-dimensional example again):

\begin{equation}
 Q = \sigma_1^a \sigma_2^{a,a} Q_{a,a} +
     \sigma_1^b \sigma_2^{b,a} Q_{b,a} +
     \sigma_1^a \sigma_2^{a,b} Q_{a,b} +
     \sigma_1^b \sigma_2^{b,b} Q_{b,b}
\end{equation}


\bibliographystyle{plain}
\bibliography{nsd}

\end{document}
