%%%%%%%%%%%%%%%%%%%%%%%%%%%%%%%%%%%%%%%%%%%%%%%%%%%%%%%%%%%%%%%%%%%%%%%%%%%
%%                           Class Description                           %%
%%%%%%%%%%%%%%%%%%%%%%%%%%%%%%%%%%%%%%%%%%%%%%%%%%%%%%%%%%%%%%%%%%%%%%%%%%%

    \index{PotentialFactory \textit{(module)}!PotentialFactory \textit{(class)}|(}
\section{Class PotentialFactory}

    \label{PotentialFactory:PotentialFactory}
A factory for \textit{MatrixPotential} instances. We decide which subclass
of the abstract base class \textit{MatrixPotential} to instantiate
according to the size of the potential's matrix. For a $1 \times 1$
matrix we can use the class \textit{MatrixPotential1S} which implements
simplified scalar symbolic calculations. In the case of a $2 \times
2$ matrix we use the class \textit{MatrixPotential2S} that implements the
full symbolic calculations for matrices. And for matrices of size bigger
than $2 \times 2$ symbolic calculations are unfeasible and we have to fall
back to pure numerical methods implemented in \textit{MatrixPotentialMS}.


%%%%%%%%%%%%%%%%%%%%%%%%%%%%%%%%%%%%%%%%%%%%%%%%%%%%%%%%%%%%%%%%%%%%%%%%%%%
%%                                Methods                                %%
%%%%%%%%%%%%%%%%%%%%%%%%%%%%%%%%%%%%%%%%%%%%%%%%%%%%%%%%%%%%%%%%%%%%%%%%%%%

  \subsection{Methods}

    \label{PotentialFactory:PotentialFactory:create_potential}
    \index{PotentialFactory \textit{(module)}!PotentialFactory \textit{(class)}!PotentialFactory.create\_potential \textit{(static method)}}

    \vspace{0.5ex}

\hspace{.8\funcindent}\begin{boxedminipage}{\funcwidth}

    \raggedright \textbf{create\_potential}(\textit{potential\_expression})

    \vspace{-1.5ex}

    \rule{\textwidth}{0.5\fboxrule}
\setlength{\parskip}{2ex}
    Static method that creates a \textit{MatrixPotential} instance and
    decides which subclass to instantiate depending on the given potential
    expression.

\setlength{\parskip}{1ex}
      \textbf{Parameters}
      \vspace{-1ex}

      \begin{quote}
        \begin{Ventry}{xxxxxxxxxxxxxxxxxxxx}

          \item[potential\_expression]

          The symbolic potential matrix given.

        \end{Ventry}

      \end{quote}

      \textbf{Return Value}
    \vspace{-1ex}

      \begin{quote}
      An adequate \textit{MatrixPotential} instance.

      \end{quote}

      \textbf{Raises}
    \vspace{-1ex}

      \begin{quote}
        \begin{description}

          \item[\texttt{ValueError}]

          If the potential matrix is not square.

        \end{description}

      \end{quote}

    \end{boxedminipage}

    \index{PotentialFactory \textit{(module)}!PotentialFactory \textit{(class)}|)}
