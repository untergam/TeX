%%%%%%%%%%%%%%%%%%%%%%%%%%%%%%%%%%%%%%%%%%%%%%%%%%%%%%%%%%%%%%%%%%%%%%%%%%%
%%                           Class Description                           %%
%%%%%%%%%%%%%%%%%%%%%%%%%%%%%%%%%%%%%%%%%%%%%%%%%%%%%%%%%%%%%%%%%%%%%%%%%%%

    \index{MatrixPotential \textit{(module)}!MatrixPotential \textit{(class)}|(}
\section{Class MatrixPotential}

    \label{MatrixPotential:MatrixPotential}
\textbf{Subclasses:}
MatrixPotential1S,
MatrixPotential2S,
MatrixPotentialMS

This class represents a potential $V\ofs{x}$. The
potential is given as an analytical expression. Some calculations with the
potential are supported. For example calculation of eigenvalues and
exponentials and numerical evaluation. Further, there are methods for
splitting the potential into a Taylor expansion and for basis
transformations between canonical and eigenbasis.


%%%%%%%%%%%%%%%%%%%%%%%%%%%%%%%%%%%%%%%%%%%%%%%%%%%%%%%%%%%%%%%%%%%%%%%%%%%
%%                                Methods                                %%
%%%%%%%%%%%%%%%%%%%%%%%%%%%%%%%%%%%%%%%%%%%%%%%%%%%%%%%%%%%%%%%%%%%%%%%%%%%

  \subsection{Methods}

    \label{MatrixPotential:MatrixPotential:__init__}
    \index{MatrixPotential \textit{(module)}!MatrixPotential \textit{(class)}!MatrixPotential.\_\_init\_\_ \textit{(method)}}

    \vspace{0.5ex}

\hspace{.8\funcindent}\begin{boxedminipage}{\funcwidth}

    \raggedright \textbf{\_\_init\_\_}(\textit{self})

    \vspace{-1.5ex}

    \rule{\textwidth}{0.5\fboxrule}
\setlength{\parskip}{2ex}
    Create a new \textit{MatrixPotential} instance for a given potential
    matrix $V\ofs{x}$.

\setlength{\parskip}{1ex}
      \textbf{Raises}
    \vspace{-1ex}

      \begin{quote}
        \begin{description}

          \item[\texttt{NotImplementedError}]

          This is an abstract base class.

        \end{description}

      \end{quote}

    \end{boxedminipage}

    \label{MatrixPotential:MatrixPotential:__str__}
    \index{MatrixPotential \textit{(module)}!MatrixPotential \textit{(class)}!MatrixPotential.\_\_str\_\_ \textit{(method)}}

    \vspace{0.5ex}

\hspace{.8\funcindent}\begin{boxedminipage}{\funcwidth}

    \raggedright \textbf{\_\_str\_\_}(\textit{self})

    \vspace{-1.5ex}

    \rule{\textwidth}{0.5\fboxrule}
\setlength{\parskip}{2ex}
    Put the number of components and the analytical expression (the matrix)
    into a printable string.

\setlength{\parskip}{1ex}
      \textbf{Raises}
    \vspace{-1ex}

      \begin{quote}
        \begin{description}

          \item[\texttt{NotImplementedError}]

          This is an abstract base class.

        \end{description}

      \end{quote}

    \end{boxedminipage}

    \label{MatrixPotential:MatrixPotential:get_number_components}
    \index{MatrixPotential \textit{(module)}!MatrixPotential \textit{(class)}!MatrixPotential.get\_number\_components \textit{(method)}}

    \vspace{0.5ex}

\hspace{.8\funcindent}\begin{boxedminipage}{\funcwidth}

    \raggedright \textbf{get\_number\_components}(\textit{self})

    \vspace{-1.5ex}

    \rule{\textwidth}{0.5\fboxrule}
\setlength{\parskip}{2ex}
\setlength{\parskip}{1ex}
      \textbf{Return Value}
    \vspace{-1ex}

      \begin{quote}
      The number $N$ of components the potential supports.

      \end{quote}

      \textbf{Raises}
    \vspace{-1ex}

      \begin{quote}
        \begin{description}

          \item[\texttt{NotImplementedError}]

          This is an abstract base class.

        \end{description}

      \end{quote}

    \end{boxedminipage}

    \label{MatrixPotential:MatrixPotential:evaluate_at}
    \index{MatrixPotential \textit{(module)}!MatrixPotential \textit{(class)}!MatrixPotential.evaluate\_at \textit{(method)}}

    \vspace{0.5ex}

\hspace{.8\funcindent}\begin{boxedminipage}{\funcwidth}

    \raggedright \textbf{evaluate\_at}(\textit{self}, \textit{nodes}, \textit{component}={\tt None})

    \vspace{-1.5ex}

    \rule{\textwidth}{0.5\fboxrule}
\setlength{\parskip}{2ex}
    Evaluate the potential matrix elementwise at some given grid nodes
    $\gamma$.

\setlength{\parskip}{1ex}
      \textbf{Parameters}
      \vspace{-1ex}

      \begin{quote}
        \begin{Ventry}{xxxxxxxxx}

          \item[nodes]

          The grid nodes $\gamma$ we want to evaluate the
          potential at.

          \item[component]

          The component $V_{i,j}$ that gets evaluated or \textit{None} to
          evaluate all.

        \end{Ventry}

      \end{quote}

      \textbf{Raises}
    \vspace{-1ex}

      \begin{quote}
        \begin{description}

          \item[\texttt{NotImplementedError}]

          This is an abstract base class.

        \end{description}

      \end{quote}

    \end{boxedminipage}

    \label{MatrixPotential:MatrixPotential:calculate_eigenvalues}
    \index{MatrixPotential \textit{(module)}!MatrixPotential \textit{(class)}!MatrixPotential.calculate\_eigenvalues \textit{(method)}}

    \vspace{0.5ex}

\hspace{.8\funcindent}\begin{boxedminipage}{\funcwidth}

    \raggedright \textbf{calculate\_eigenvalues}(\textit{self})

    \vspace{-1.5ex}

    \rule{\textwidth}{0.5\fboxrule}
\setlength{\parskip}{2ex}
    Calculate the eigenvalues
    $\lambda_i\ofs{x}$ of the potential
    $V\ofs{x}$.

\setlength{\parskip}{1ex}
      \textbf{Raises}
    \vspace{-1ex}

      \begin{quote}
        \begin{description}

          \item[\texttt{NotImplementedError}]

          This is an abstract base class.

        \end{description}

      \end{quote}

    \end{boxedminipage}

    \label{MatrixPotential:MatrixPotential:evaluate_eigenvalues_at}
    \index{MatrixPotential \textit{(module)}!MatrixPotential \textit{(class)}!MatrixPotential.evaluate\_eigenvalues\_at \textit{(method)}}

    \vspace{0.5ex}

\hspace{.8\funcindent}\begin{boxedminipage}{\funcwidth}

    \raggedright \textbf{evaluate\_eigenvalues\_at}(\textit{self}, \textit{nodes}, \textit{diagonal\_component}={\tt None})

    \vspace{-1.5ex}

    \rule{\textwidth}{0.5\fboxrule}
\setlength{\parskip}{2ex}
    Evaluate the eigenvalues
    $\lambda_i\ofs{x}$ at some grid
    nodes $\gamma$.

\setlength{\parskip}{1ex}
      \textbf{Parameters}
      \vspace{-1ex}

      \begin{quote}
        \begin{Ventry}{xxxxxxxxxxxxxxxxxx}

          \item[nodes]

          The grid nodes $\gamma$ we want to evaluate the
          eigenvalues at.

          \item[diagonal\_component]

          The index $i$ of the eigenvalue $\lambda_i$
          that gets evaluated or \textit{None} to evaluate all.

        \end{Ventry}

      \end{quote}

      \textbf{Raises}
    \vspace{-1ex}

      \begin{quote}
        \begin{description}

          \item[\texttt{NotImplementedError}]

          This is an abstract base class.

        \end{description}

      \end{quote}

    \end{boxedminipage}

    \label{MatrixPotential:MatrixPotential:calculate_eigenvectors}
    \index{MatrixPotential \textit{(module)}!MatrixPotential \textit{(class)}!MatrixPotential.calculate\_eigenvectors \textit{(method)}}

    \vspace{0.5ex}

\hspace{.8\funcindent}\begin{boxedminipage}{\funcwidth}

    \raggedright \textbf{calculate\_eigenvectors}(\textit{self})

    \vspace{-1.5ex}

    \rule{\textwidth}{0.5\fboxrule}
\setlength{\parskip}{2ex}
    Calculate the eigenvectors $\nu_i\ofs{x}$ of the
    potential $V\ofs{x}$.

\setlength{\parskip}{1ex}
      \textbf{Raises}
    \vspace{-1ex}

      \begin{quote}
        \begin{description}

          \item[\texttt{NotImplementedError}]

          This is an abstract base class.

        \end{description}

      \end{quote}

    \end{boxedminipage}

    \label{MatrixPotential:MatrixPotential:evaluate_eigenvectors_at}
    \index{MatrixPotential \textit{(module)}!MatrixPotential \textit{(class)}!MatrixPotential.evaluate\_eigenvectors\_at \textit{(method)}}

    \vspace{0.5ex}

\hspace{.8\funcindent}\begin{boxedminipage}{\funcwidth}

    \raggedright \textbf{evaluate\_eigenvectors\_at}(\textit{self}, \textit{nodes})

    \vspace{-1.5ex}

    \rule{\textwidth}{0.5\fboxrule}
\setlength{\parskip}{2ex}
    Evaluate the eigenvectors $\nu_i\ofs{x}$ at some
    grid nodes $\gamma$.

\setlength{\parskip}{1ex}
      \textbf{Parameters}
      \vspace{-1ex}

      \begin{quote}
        \begin{Ventry}{xxxxx}

          \item[nodes]

          The grid nodes $\gamma$ we want to evaluate the
          eigenvectors at.

        \end{Ventry}

      \end{quote}

      \textbf{Raises}
    \vspace{-1ex}

      \begin{quote}
        \begin{description}

          \item[\texttt{NotImplementedError}]

          This is an abstract base class.

        \end{description}

      \end{quote}

    \end{boxedminipage}

    \label{MatrixPotential:MatrixPotential:project_to_eigen}
    \index{MatrixPotential \textit{(module)}!MatrixPotential \textit{(class)}!MatrixPotential.project\_to\_eigen \textit{(method)}}

    \vspace{0.5ex}

\hspace{.8\funcindent}\begin{boxedminipage}{\funcwidth}

    \raggedright \textbf{project\_to\_eigen}(\textit{self}, \textit{nodes}, \textit{values}, \textit{basis}={\tt None})

    \vspace{-1.5ex}

    \rule{\textwidth}{0.5\fboxrule}
\setlength{\parskip}{2ex}
    Project a given vector from the canonical basis to the eigenbasis of
    the potential.

\setlength{\parskip}{1ex}
      \textbf{Parameters}
      \vspace{-1ex}

      \begin{quote}
        \begin{Ventry}{xxxxxx}

          \item[nodes]

          The grid nodes $\gamma$ for the pointwise
          transformation.

          \item[values]

          The list of vectors $\varphi_i$ containing the
          values we want to transform.

          \item[basis]

          A list of basisvectors $\nu_i$. Allows to use this function for
          external data, similar to a static function.

        \end{Ventry}

      \end{quote}

      \textbf{Raises}
    \vspace{-1ex}

      \begin{quote}
        \begin{description}

          \item[\texttt{NotImplementedError}]

          This is an abstract base class.

        \end{description}

      \end{quote}

    \end{boxedminipage}

    \label{MatrixPotential:MatrixPotential:project_to_canonical}
    \index{MatrixPotential \textit{(module)}!MatrixPotential \textit{(class)}!MatrixPotential.project\_to\_canonical \textit{(method)}}

    \vspace{0.5ex}

\hspace{.8\funcindent}\begin{boxedminipage}{\funcwidth}

    \raggedright \textbf{project\_to\_canonical}(\textit{self}, \textit{nodes}, \textit{values}, \textit{basis}={\tt None})

    \vspace{-1.5ex}

    \rule{\textwidth}{0.5\fboxrule}
\setlength{\parskip}{2ex}
    Project a given vector from the potential's eigenbasis to the canonical
    basis.

\setlength{\parskip}{1ex}
      \textbf{Parameters}
      \vspace{-1ex}

      \begin{quote}
        \begin{Ventry}{xxxxxx}

          \item[nodes]

          The grid nodes $\gamma$ for the pointwise
          transformation.

          \item[values]

          The list of vectors $\varphi_i$ containing the values we want
          to transform.

          \item[basis]

          A list of basis vectors $\nu_i$. Allows to use this function
          for external data, similar to a static function.

        \end{Ventry}

      \end{quote}

      \textbf{Raises}
    \vspace{-1ex}

      \begin{quote}
        \begin{description}

          \item[\texttt{NotImplementedError}]

          This is an abstract base class.

        \end{description}

      \end{quote}

    \end{boxedminipage}

    \label{MatrixPotential:MatrixPotential:calculate_exponential}
    \index{MatrixPotential \textit{(module)}!MatrixPotential \textit{(class)}!MatrixPotential.calculate\_exponential \textit{(method)}}

    \vspace{0.5ex}

\hspace{.8\funcindent}\begin{boxedminipage}{\funcwidth}

    \raggedright \textbf{calculate\_exponential}(\textit{self}, \textit{factor}={\tt 1})

    \vspace{-1.5ex}

    \rule{\textwidth}{0.5\fboxrule}
\setlength{\parskip}{2ex}
    Calculate the matrix exponential $E =
    \exp\ofs{\alpha M}$.

\setlength{\parskip}{1ex}
      \textbf{Parameters}
      \vspace{-1ex}

      \begin{quote}
        \begin{Ventry}{xxxxxx}

          \item[factor]

          A prefactor $\alpha$ in the exponential.

        \end{Ventry}

      \end{quote}

      \textbf{Raises}
    \vspace{-1ex}

      \begin{quote}
        \begin{description}

          \item[\texttt{NotImplementedError}]

          This is an abstract base class.

        \end{description}

      \end{quote}

    \end{boxedminipage}

    \label{MatrixPotential:MatrixPotential:evaluate_exponential_at}
    \index{MatrixPotential \textit{(module)}!MatrixPotential \textit{(class)}!MatrixPotential.evaluate\_exponential\_at \textit{(method)}}

    \vspace{0.5ex}

\hspace{.8\funcindent}\begin{boxedminipage}{\funcwidth}

    \raggedright \textbf{evaluate\_exponential\_at}(\textit{self}, \textit{nodes})

    \vspace{-1.5ex}

    \rule{\textwidth}{0.5\fboxrule}
\setlength{\parskip}{2ex}
    Evaluate the exponential of the potential matrix $V$ at some grid
    nodes $\gamma$.

\setlength{\parskip}{1ex}
      \textbf{Parameters}
      \vspace{-1ex}

      \begin{quote}
        \begin{Ventry}{xxxxx}

          \item[nodes]

          The grid nodes $\gamma$ we want to evaluate the
          exponential at.

        \end{Ventry}

      \end{quote}

      \textbf{Raises}
    \vspace{-1ex}

      \begin{quote}
        \begin{description}

          \item[\texttt{NotImplementedError}]

          This is an abstract base class.

        \end{description}

      \end{quote}

    \end{boxedminipage}

    \label{MatrixPotential:MatrixPotential:calculate_jacobian}
    \index{MatrixPotential \textit{(module)}!MatrixPotential \textit{(class)}!MatrixPotential.calculate\_jacobian \textit{(method)}}

    \vspace{0.5ex}

\hspace{.8\funcindent}\begin{boxedminipage}{\funcwidth}

    \raggedright \textbf{calculate\_jacobian}(\textit{self})

    \vspace{-1.5ex}

    \rule{\textwidth}{0.5\fboxrule}
\setlength{\parskip}{2ex}
    Calculate the Jacobian matrix for each component $V_{i,j}$ of the
    potential. For potentials which depend only one variable $x$, this
    equals the first derivative.

\setlength{\parskip}{1ex}
      \textbf{Raises}
    \vspace{-1ex}

      \begin{quote}
        \begin{description}

          \item[\texttt{NotImplementedError}]

          This is an abstract base class.

        \end{description}

      \end{quote}

    \end{boxedminipage}

    \label{MatrixPotential:MatrixPotential:evaluate_jacobian_at}
    \index{MatrixPotential \textit{(module)}!MatrixPotential \textit{(class)}!MatrixPotential.evaluate\_jacobian\_at \textit{(method)}}

    \vspace{0.5ex}

\hspace{.8\funcindent}\begin{boxedminipage}{\funcwidth}

    \raggedright \textbf{evaluate\_jacobian\_at}(\textit{self}, \textit{nodes}, \textit{component}={\tt None})

    \vspace{-1.5ex}

    \rule{\textwidth}{0.5\fboxrule}
\setlength{\parskip}{2ex}
    Evaluate the Jacobian at some grid nodes $\gamma$ for
    each component $V_{i,j}$ of the potential.

\setlength{\parskip}{1ex}
      \textbf{Parameters}
      \vspace{-1ex}

      \begin{quote}
        \begin{Ventry}{xxxxxxxxx}

          \item[nodes]

          The grid nodes $\gamma$ the Jacobian gets
          evaluated at.

          \item[component]

          The index tuple $\left(i,j \right)$ that specifies
          the potential's entry of which the Jacobian is evaluated.
          (Defaults to \textit{None} to evaluate all)

        \end{Ventry}

      \end{quote}

      \textbf{Raises}
    \vspace{-1ex}

      \begin{quote}
        \begin{description}

          \item[\texttt{NotImplementedError}]

          This is an abstract base class.

        \end{description}

      \end{quote}

    \end{boxedminipage}

    \label{MatrixPotential:MatrixPotential:calculate_hessian}
    \index{MatrixPotential \textit{(module)}!MatrixPotential \textit{(class)}!MatrixPotential.calculate\_hessian \textit{(method)}}

    \vspace{0.5ex}

\hspace{.8\funcindent}\begin{boxedminipage}{\funcwidth}

    \raggedright \textbf{calculate\_hessian}(\textit{self})

    \vspace{-1.5ex}

    \rule{\textwidth}{0.5\fboxrule}
\setlength{\parskip}{2ex}
    Calculate the Hessian matrix for each component $V_{i,j}$ of the
    potential. For potentials which depend only one variable $x$, this
    equals the second derivative.

\setlength{\parskip}{1ex}
      \textbf{Raises}
    \vspace{-1ex}

      \begin{quote}
        \begin{description}

          \item[\texttt{NotImplementedError}]

          This is an abstract base class.

        \end{description}

      \end{quote}

    \end{boxedminipage}

    \label{MatrixPotential:MatrixPotential:evaluate_hessian_at}
    \index{MatrixPotential \textit{(module)}!MatrixPotential \textit{(class)}!MatrixPotential.evaluate\_hessian\_at \textit{(method)}}

    \vspace{0.5ex}

\hspace{.8\funcindent}\begin{boxedminipage}{\funcwidth}

    \raggedright \textbf{evaluate\_hessian\_at}(\textit{self}, \textit{nodes}, \textit{component}={\tt None})

    \vspace{-1.5ex}

    \rule{\textwidth}{0.5\fboxrule}
\setlength{\parskip}{2ex}
    Evaluate the Hessian at some grid nodes $\gamma$ for
    each component $V_{i,j}$ of the potential.

\setlength{\parskip}{1ex}
      \textbf{Parameters}
      \vspace{-1ex}

      \begin{quote}
        \begin{Ventry}{xxxxxxxxx}

          \item[nodes]

          The grid nodes $\gamma$ the Hessian gets
          evaluated at.

          \item[component]

          The index tuple $\left(i,j \right)$ that specifies
          the potential's entry of which the Hessian is evaluated. (Or
          \textit{None} to evaluate all)

        \end{Ventry}

      \end{quote}

      \textbf{Raises}
    \vspace{-1ex}

      \begin{quote}
        \begin{description}

          \item[\texttt{NotImplementedError}]

          This is an abstract base class.

        \end{description}

      \end{quote}

    \end{boxedminipage}

    \label{MatrixPotential:MatrixPotential:calculate_local_quadratic}
    \index{MatrixPotential \textit{(module)}!MatrixPotential \textit{(class)}!MatrixPotential.calculate\_local\_quadratic \textit{(method)}}

    \vspace{0.5ex}

\hspace{.8\funcindent}\begin{boxedminipage}{\funcwidth}

    \raggedright \textbf{calculate\_local\_quadratic}(\textit{self}, \textit{diagonal\_component}={\tt None})

    \vspace{-1.5ex}

    \rule{\textwidth}{0.5\fboxrule}
\setlength{\parskip}{2ex}
    Calculate the local quadratic approximation matrix $U$ of the
    potential's eigenvalues in $\Lambda$. This function is
    used for the homogeneous case and takes into account the leading
    component $\chi$.

\setlength{\parskip}{1ex}
      \textbf{Parameters}
      \vspace{-1ex}

      \begin{quote}
        \begin{Ventry}{xxxxxxxxxxxxxxxxxx}

          \item[diagonal\_component]

          Specifies the index $i$ of the eigenvalue
          $\lambda_i$ that gets expanded into a Taylor
          series $u_i$.

        \end{Ventry}

      \end{quote}

      \textbf{Raises}
    \vspace{-1ex}

      \begin{quote}
        \begin{description}

          \item[\texttt{NotImplementedError}]

          This is an abstract base class.

        \end{description}

      \end{quote}

    \end{boxedminipage}

    \label{MatrixPotential:MatrixPotential:evaluate_local_quadratic_at}
    \index{MatrixPotential \textit{(module)}!MatrixPotential \textit{(class)}!MatrixPotential.evaluate\_local\_quadratic\_at \textit{(method)}}

    \vspace{0.5ex}

\hspace{.8\funcindent}\begin{boxedminipage}{\funcwidth}

    \raggedright \textbf{evaluate\_local\_quadratic\_at}(\textit{self}, \textit{nodes})

    \vspace{-1.5ex}

    \rule{\textwidth}{0.5\fboxrule}
\setlength{\parskip}{2ex}
    Numerically evaluate the local quadratic approximation matrix $U$ of
    the potential's eigenvalues in $\Lambda$ at the given
    grid nodes $\gamma$. This function is used for the
    homogeneous case and takes into account the leading component
    $\chi$.

\setlength{\parskip}{1ex}
      \textbf{Parameters}
      \vspace{-1ex}

      \begin{quote}
        \begin{Ventry}{xxxxx}

          \item[nodes]

          The grid nodes $\gamma$ we want to evaluate the
          quadratic approximation at.

        \end{Ventry}

      \end{quote}

      \textbf{Raises}
    \vspace{-1ex}

      \begin{quote}
        \begin{description}

          \item[\texttt{NotImplementedError}]

          This is an abstract base class.

        \end{description}

      \end{quote}

    \end{boxedminipage}

    \label{MatrixPotential:MatrixPotential:calculate_local_remainder}
    \index{MatrixPotential \textit{(module)}!MatrixPotential \textit{(class)}!MatrixPotential.calculate\_local\_remainder \textit{(method)}}

    \vspace{0.5ex}

\hspace{.8\funcindent}\begin{boxedminipage}{\funcwidth}

    \raggedright \textbf{calculate\_local\_remainder}(\textit{self}, \textit{diagonal\_component}={\tt 0})

    \vspace{-1.5ex}

    \rule{\textwidth}{0.5\fboxrule}
\setlength{\parskip}{2ex}
    Calculate the non-quadratic remainder matrix $W$ of the quadratic
    approximation matrix $U$ of the potential's eigenvalue matrix
    $\Lambda$. This function is used for the homogeneous
    case and takes into account the leading component
    $\chi$.

\setlength{\parskip}{1ex}
      \textbf{Parameters}
      \vspace{-1ex}

      \begin{quote}
        \begin{Ventry}{xxxxxxxxxxxxxxxxxx}

          \item[diagonal\_component]

          Specifies the index $\chi$ of the leading
          component $\lambda_\chi$.

        \end{Ventry}

      \end{quote}

      \textbf{Raises}
    \vspace{-1ex}

      \begin{quote}
        \begin{description}

          \item[\texttt{NotImplementedError}]

          This is an abstract base class.

        \end{description}

      \end{quote}

    \end{boxedminipage}

    \label{MatrixPotential:MatrixPotential:evaluate_local_remainder_at}
    \index{MatrixPotential \textit{(module)}!MatrixPotential \textit{(class)}!MatrixPotential.evaluate\_local\_remainder\_at \textit{(method)}}

    \vspace{0.5ex}

\hspace{.8\funcindent}\begin{boxedminipage}{\funcwidth}

    \raggedright \textbf{evaluate\_local\_remainder\_at}(\textit{self}, \textit{position}, \textit{nodes}, \textit{component}={\tt None})

    \vspace{-1.5ex}

    \rule{\textwidth}{0.5\fboxrule}
\setlength{\parskip}{2ex}
    Numerically evaluate the non-quadratic remainder matrix $W$ of the
    quadratic approximation matrix $U$ of the potential's eigenvalues in
    $\Lambda$ at the given nodes
    $\gamma$. This function is used for the homogeneous
    and the inhomogeneous case and just evaluates the remainder matrix
    $W$.

\setlength{\parskip}{1ex}
      \textbf{Parameters}
      \vspace{-1ex}

      \begin{quote}
        \begin{Ventry}{xxxxxxxxx}

          \item[position]

          The point $q$ where the Taylor series is computed.

          \item[nodes]

          The grid nodes $\gamma$ we want to evaluate the
          potential at.

          \item[component]

          The component $\left(i,j \right)$ of the remainder
          matrix $W$ that is evaluated.

        \end{Ventry}

      \end{quote}

      \textbf{Raises}
    \vspace{-1ex}

      \begin{quote}
        \begin{description}

          \item[\texttt{NotImplementedError}]

          This is an abstract base class.

        \end{description}

      \end{quote}

    \end{boxedminipage}

    \label{MatrixPotential:MatrixPotential:calculate_local_quadratic_multi}
    \index{MatrixPotential \textit{(module)}!MatrixPotential \textit{(class)}!MatrixPotential.calculate\_local\_quadratic\_multi \textit{(method)}}

    \vspace{0.5ex}

\hspace{.8\funcindent}\begin{boxedminipage}{\funcwidth}

    \raggedright \textbf{calculate\_local\_quadratic\_multi}(\textit{self})

    \vspace{-1.5ex}

    \rule{\textwidth}{0.5\fboxrule}
\setlength{\parskip}{2ex}
    Calculate the local quadratic approximation matrix $U$ of all the
    potential's eigenvalues in $\Lambda$. This function is
    used for the inhomogeneous case.

\setlength{\parskip}{1ex}
      \textbf{Raises}
    \vspace{-1ex}

      \begin{quote}
        \begin{description}

          \item[\texttt{NotImplementedError}]

          This is an abstract base class.

        \end{description}

      \end{quote}

    \end{boxedminipage}

    \label{MatrixPotential:MatrixPotential:evaluate_local_quadratic_multi_at}
    \index{MatrixPotential \textit{(module)}!MatrixPotential \textit{(class)}!MatrixPotential.evaluate\_local\_quadratic\_multi\_at \textit{(method)}}

    \vspace{0.5ex}

\hspace{.8\funcindent}\begin{boxedminipage}{\funcwidth}

    \raggedright \textbf{evaluate\_local\_quadratic\_multi\_at}(\textit{self}, \textit{nodes}, \textit{component}={\tt None})

    \vspace{-1.5ex}

    \rule{\textwidth}{0.5\fboxrule}
\setlength{\parskip}{2ex}
    Numerically evaluate the local quadratic approximation matrix $U$ of
    the potential's eigenvalues in $\Lambda$ at the given
    grid nodes $\gamma$. This function is used for the
    inhomogeneous case.

\setlength{\parskip}{1ex}
      \textbf{Parameters}
      \vspace{-1ex}

      \begin{quote}
        \begin{Ventry}{xxxxxxxxx}

          \item[nodes]

          The grid nodes $\gamma$ we want to evaluate the
          quadratic approximation at.

          \item[component]

          The component $\left(i,j \right)$ of the quadratic
          approximation matrix $U$ that is evaluated.

        \end{Ventry}

      \end{quote}

      \textbf{Raises}
    \vspace{-1ex}

      \begin{quote}
        \begin{description}

          \item[\texttt{NotImplementedError}]

          This is an abstract base class.

        \end{description}

      \end{quote}

    \end{boxedminipage}

    \label{MatrixPotential:MatrixPotential:calculate_local_remainder_multi}
    \index{MatrixPotential \textit{(module)}!MatrixPotential \textit{(class)}!MatrixPotential.calculate\_local\_remainder\_multi \textit{(method)}}

    \vspace{0.5ex}

\hspace{.8\funcindent}\begin{boxedminipage}{\funcwidth}

    \raggedright \textbf{calculate\_local\_remainder\_multi}(\textit{self})

    \vspace{-1.5ex}

    \rule{\textwidth}{0.5\fboxrule}
\setlength{\parskip}{2ex}
    Calculate the non-quadratic remainder matrix $W$ of the quadratic
    approximation matrix $U$ of the potential's eigenvalue matrix
    $\Lambda$. This function is used for the inhomogeneous
    case.

\setlength{\parskip}{1ex}
      \textbf{Raises}
    \vspace{-1ex}

      \begin{quote}
        \begin{description}

          \item[\texttt{NotImplementedError}]

          This is an abstract base class.

        \end{description}

      \end{quote}

    \end{boxedminipage}

    \index{MatrixPotential \textit{(module)}!MatrixPotential \textit{(class)}|)}
