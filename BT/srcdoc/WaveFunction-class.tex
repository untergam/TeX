%%%%%%%%%%%%%%%%%%%%%%%%%%%%%%%%%%%%%%%%%%%%%%%%%%%%%%%%%%%%%%%%%%%%%%%%%%%
%%                           Class Description                           %%
%%%%%%%%%%%%%%%%%%%%%%%%%%%%%%%%%%%%%%%%%%%%%%%%%%%%%%%%%%%%%%%%%%%%%%%%%%%

    \index{WaveFunction \textit{(module)}!WaveFunction \textit{(class)}|(}
\section{Class WaveFunction}

    \label{WaveFunction:WaveFunction}
This class represents a vector valued quantum state
$\Ket{\Psi}$ as used in the vector valued
time-dependent Schroedinger equation. The state
$\Ket{\Psi}$ is composed of
$\psi_0, \ldots,
\psi_{N-1}$ where $\psi_i$ is a single
wavefunction component.


%%%%%%%%%%%%%%%%%%%%%%%%%%%%%%%%%%%%%%%%%%%%%%%%%%%%%%%%%%%%%%%%%%%%%%%%%%%
%%                                Methods                                %%
%%%%%%%%%%%%%%%%%%%%%%%%%%%%%%%%%%%%%%%%%%%%%%%%%%%%%%%%%%%%%%%%%%%%%%%%%%%

  \subsection{Methods}

    \label{WaveFunction:WaveFunction:__init__}
    \index{WaveFunction \textit{(module)}!WaveFunction \textit{(class)}!WaveFunction.\_\_init\_\_ \textit{(method)}}

    \vspace{0.5ex}

\hspace{.8\funcindent}\begin{boxedminipage}{\funcwidth}

    \raggedright \textbf{\_\_init\_\_}(\textit{self}, \textit{nodes}, \textit{values})

    \vspace{-1.5ex}

    \rule{\textwidth}{0.5\fboxrule}
\setlength{\parskip}{2ex}
    Initialize the \textit{WaveFunction} object that represents the vector
    of states $\Ket{\Psi}$.

\setlength{\parskip}{1ex}
      \textbf{Parameters}
      \vspace{-1ex}

      \begin{quote}
        \begin{Ventry}{xxxxxx}

          \item[nodes]

          The grid nodes to which the numerical values of
          $\psi_i$ belong to.

          \item[values]

          A list with the numerical values of each component
          $\psi_i$ sampled at the given nodes.

        \end{Ventry}

      \end{quote}

    \end{boxedminipage}

    \label{WaveFunction:WaveFunction:__str__}
    \index{WaveFunction \textit{(module)}!WaveFunction \textit{(class)}!WaveFunction.\_\_str\_\_ \textit{(method)}}

    \vspace{0.5ex}

\hspace{.8\funcindent}\begin{boxedminipage}{\funcwidth}

    \raggedright \textbf{\_\_str\_\_}(\textit{self})

    \vspace{-1.5ex}

    \rule{\textwidth}{0.5\fboxrule}
\setlength{\parskip}{2ex}
\setlength{\parskip}{1ex}
      \textbf{Return Value}
    \vspace{-1ex}

      \begin{quote}
      A string that describes the wavefunction
      $\Ket{\Psi}$.

      \end{quote}

    \end{boxedminipage}

    \label{WaveFunction:WaveFunction:get_number_components}
    \index{WaveFunction \textit{(module)}!WaveFunction \textit{(class)}!WaveFunction.get\_number\_components \textit{(method)}}

    \vspace{0.5ex}

\hspace{.8\funcindent}\begin{boxedminipage}{\funcwidth}

    \raggedright \textbf{get\_number\_components}(\textit{self})

    \vspace{-1.5ex}

    \rule{\textwidth}{0.5\fboxrule}
\setlength{\parskip}{2ex}
\setlength{\parskip}{1ex}
      \textbf{Return Value}
    \vspace{-1ex}

      \begin{quote}
      The number of components $\psi_i$ the vector
      $\Ket{\Psi}$ consists of.

      \end{quote}

    \end{boxedminipage}

    \label{WaveFunction:WaveFunction:get_nodes}
    \index{WaveFunction \textit{(module)}!WaveFunction \textit{(class)}!WaveFunction.get\_nodes \textit{(method)}}

    \vspace{0.5ex}

\hspace{.8\funcindent}\begin{boxedminipage}{\funcwidth}

    \raggedright \textbf{get\_nodes}(\textit{self})

    \vspace{-1.5ex}

    \rule{\textwidth}{0.5\fboxrule}
\setlength{\parskip}{2ex}
\setlength{\parskip}{1ex}
      \textbf{Return Value}
    \vspace{-1ex}

      \begin{quote}
      The grid nodes $\gamma$ the wave function values
      belong to.

      \end{quote}

    \end{boxedminipage}

    \label{WaveFunction:WaveFunction:get_values}
    \index{WaveFunction \textit{(module)}!WaveFunction \textit{(class)}!WaveFunction.get\_values \textit{(method)}}

    \vspace{0.5ex}

\hspace{.8\funcindent}\begin{boxedminipage}{\funcwidth}

    \raggedright \textbf{get\_values}(\textit{self})

    \vspace{-1.5ex}

    \rule{\textwidth}{0.5\fboxrule}
\setlength{\parskip}{2ex}
    Return the wave function values for each component of
    $\Ket{\Psi}$.

\setlength{\parskip}{1ex}
      \textbf{Return Value}
    \vspace{-1ex}

      \begin{quote}
      A list with the values of all components $\psi_i$
      evaluated on the grid nodes $\gamma$.

      \end{quote}

    \end{boxedminipage}

    \label{WaveFunction:WaveFunction:set_values}
    \index{WaveFunction \textit{(module)}!WaveFunction \textit{(class)}!WaveFunction.set\_values \textit{(method)}}

    \vspace{0.5ex}

\hspace{.8\funcindent}\begin{boxedminipage}{\funcwidth}

    \raggedright \textbf{set\_values}(\textit{self}, \textit{values})

    \vspace{-1.5ex}

    \rule{\textwidth}{0.5\fboxrule}
\setlength{\parskip}{2ex}
    Assign new function values for each component of
    $\Ket{\Psi}$.

\setlength{\parskip}{1ex}
      \textbf{Parameters}
      \vspace{-1ex}

      \begin{quote}
        \begin{Ventry}{xxxxxx}

          \item[values]

          A list with the new values of all the $\psi_i$.

        \end{Ventry}

      \end{quote}

      \textbf{Raises}
    \vspace{-1ex}

      \begin{quote}
        \begin{description}

          \item[\texttt{ValueError}]

          If the list \textit{values} has the wrong number of entries.

        \end{description}

      \end{quote}

    \end{boxedminipage}

    \label{WaveFunction:WaveFunction:get_norm}
    \index{WaveFunction \textit{(module)}!WaveFunction \textit{(class)}!WaveFunction.get\_norm \textit{(method)}}

    \vspace{0.5ex}

\hspace{.8\funcindent}\begin{boxedminipage}{\funcwidth}

    \raggedright \textbf{get\_norm}(\textit{self}, \textit{values}={\tt None}, \textit{summed}={\tt False}, \textit{component}={\tt None})

    \vspace{-1.5ex}

    \rule{\textwidth}{0.5\fboxrule}
\setlength{\parskip}{2ex}
    Calculate the $L^2$ norm of the whole vector
    $\Ket{\Psi}$ or some individual
    components $\psi_i$. The calculation is done in
    momentum space.

\setlength{\parskip}{1ex}
      \textbf{Parameters}
      \vspace{-1ex}

      \begin{quote}
        \begin{Ventry}{xxxxxxxxx}

          \item[values]

          Allows to use this function for external data, similar to a
          static function.

          \item[summed]

          Whether to sum up the norms of the individual components.

          \item[component]

          The component $\psi_i$ of which the norm is
          calculated.

        \end{Ventry}

      \end{quote}

      \textbf{Return Value}
    \vspace{-1ex}

      \begin{quote}
      The $L^2$ norm of
      $\Ket{\Psi}$ or a list of the
      $L^2$ norms of all components $\psi_i$. (Depending on the
      optional arguments.)

      \end{quote}

    \end{boxedminipage}

    \label{WaveFunction:WaveFunction:kinetic_energy}
    \index{WaveFunction \textit{(module)}!WaveFunction \textit{(class)}!WaveFunction.kinetic\_energy \textit{(method)}}

    \vspace{0.5ex}

\hspace{.8\funcindent}\begin{boxedminipage}{\funcwidth}

    \raggedright \textbf{kinetic\_energy}(\textit{self}, \textit{kinetic}, \textit{summed}={\tt False})

    \vspace{-1.5ex}

    \rule{\textwidth}{0.5\fboxrule}
\setlength{\parskip}{2ex}
    Calculate the kinetic energy $E_{\text{kin}} \assign
    \Braket{\Psi|T|\Psi}$
    of the different components.

\setlength{\parskip}{1ex}
      \textbf{Parameters}
      \vspace{-1ex}

      \begin{quote}
        \begin{Ventry}{xxxxxxx}

          \item[kinetic]

          The kinetic energy operator $T$.

          \item[summed]

          Whether to sum up the kinetic energies of the individual
          components.

        \end{Ventry}

      \end{quote}

      \textbf{Return Value}
    \vspace{-1ex}

      \begin{quote}
      A list with the kinetic energies of the individual components or the
      overall kinetic energy of the wavefunction. (Depending on the
      optional arguments.)

      \end{quote}

    \end{boxedminipage}

    \label{WaveFunction:WaveFunction:potential_energy}
    \index{WaveFunction \textit{(module)}!WaveFunction \textit{(class)}!WaveFunction.potential\_energy \textit{(method)}}

    \vspace{0.5ex}

\hspace{.8\funcindent}\begin{boxedminipage}{\funcwidth}

    \raggedright \textbf{potential\_energy}(\textit{self}, \textit{potential}, \textit{summed}={\tt False})

    \vspace{-1.5ex}

    \rule{\textwidth}{0.5\fboxrule}
\setlength{\parskip}{2ex}
    Calculate the potential energy $E_{\text{pot}} \assign
    \Braket{\Psi|V|\Psi}$
    of the different components.

\setlength{\parskip}{1ex}
      \textbf{Parameters}
      \vspace{-1ex}

      \begin{quote}
        \begin{Ventry}{xxxxxxxxx}

          \item[potential]

          The potential energy operator $V$.

          \item[summed]

          Whether to sum up the potential energies of the individual
          components.

        \end{Ventry}

      \end{quote}

      \textbf{Return Value}
    \vspace{-1ex}

      \begin{quote}
      A list with the potential energies of the individual components or
      the overall potential energy of the wavefunction. (Depending on the
      optional arguments.)

      \end{quote}

    \end{boxedminipage}

    \index{WaveFunction \textit{(module)}!WaveFunction \textit{(class)}|)}
