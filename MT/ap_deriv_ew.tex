\chapter{Derivatives of Eigenvalues}
\label{ch:deriv_ew}


In this small appendix we show the computation of derivatives of the eigenvalues.
Although these formulae are well known for example in structural mechanics
and similar areas of applied mathematics, we think that it is a good idea to show
them here. For more detailed comments and proofs see the references given below.

We start with a real and symmetric $N \times N$ matrix $\mat{A}$. Each
of its entries is a function $a_{r,c}$ dependent on some parameters.
In our case these are the position space variables $x_0, \ldots x_{D-1}$.
Hence we write $\mat{A}(\vec{x})$. Next we assume that the matrix is
diagonalisable:

\begin{equation}
  \mat{A} = \mat{M} \mat{\Lambda} \mat{M}\T
\end{equation}

also compare to \eqref{eq:eigentransformation}. The eigenvectors are orthogonal
by theory and we take them normalised, which we can write as:

\begin{equation}
  \mat{M}\T \mat{M} = \id \,.
\end{equation}

And finally we require the eigenvalues $\lambda(\vec{x})$ to be all distinct
for all values of the parameters $\vec{x}$. Then we can write the first derivatives
as:

\begin{equation}
  \pdiff{\lambda_i(\vec{x})}{x_j} = \vec{\nu_i}\T \, \pdiff{\mat{A}(\vec{x})}{x_j} \, \vec{\nu_i}
\end{equation}

for all $i, j \in 0, \ldots N-1$. With this formula we can compute gradients
and Jacobians for all eigenvalues. The important point is that we know the
analytic closed-form expressions for $a_{r,c}(\vec{x})$ and computing their
derivatives thus is trivial.

For the second derivatives we have a similar formula:

\begin{equation}
  \frac{\partial^2 \lambda_i(\vec{x})}{\partial x_j \partial x_k} =
  \vec{\nu_i}\T \, \frac{\partial^2 \mat{A}(\vec{x})}{\partial x_j \partial x_k} \, \vec{\nu_i}
  + 2 \sum_{l \neq i} \frac{
    \left(\vec{\nu_i}\T \, \pdiff{\mat{A}(\vec{x})}{x_j} \, \vec{\nu_l}\right)
    \left(\vec{\nu_i}\T \, \pdiff{\mat{A}(\vec{x})}{x_k} \, \vec{\nu_l}\right)
  }{\lambda_i - \lambda_l} \,.
\end{equation}

With the help of this formula we can compute Hessian matrices for all eigenvalues easily.
These formulae provide us with the tools to compute the quadratic approximation
in algorithms \ref{al:tp_wave_packets_homog} and \ref{al:tp_wave_packets_inhomog}.

For more comprehensive information on this topic see for example \cite{L_eigenvalues, MH_eigenvalues, OW_eigenvalues}.
